\documentclass[12pt]{article}
\usepackage[left=3.5cm,top=3.7cm,right=3.5cm,bottom=3.5cm]{geometry}
\usepackage[utf8]{inputenc}
\usepackage[T1]{fontenc}
\usepackage[finnish]{babel}
\usepackage{ebgaramond}
\usepackage{setspace}
\setcounter{secnumdepth}{0}

\title{A TEMPORARY STRUGGLE \\ \vspace{4 mm} {\normalsize BRIEF NOTES ON THE GAME OVERPOPULATION}}
\author{As composed by T. K. Oih on behalf of L. T. P.}
\date{}

\begin{document}
\maketitle
\thispagestyle{empty}
\onehalfspacing
\section{}As I was requested by my beloved comrade J. K. Giih to write either a preface or a postscript for his latest personal computer game, I felt a nervous wave of uncertainty penetrating the delicate walls of my astral body, leaving me in a state of disheartening stagnation. Who am I, after all, to cast my undoubtedly biased interpretation upon this minimalist experience which in itself does not seem to state anything very specific about either the world or the human condition, let alone argue for particular philosophical viewpoints?

First of all, when I dived blind into the game, I could not even grasp what the goal of it might be. (And yet, is this not also the case with life itself?) When addressed, the author kindly revealed that the game can be ``won'' by reuniting the four fellows torn apart and confused by the insanity running rampant in human civilizations of today and subsequently controlled with the W, A, S, D -keys.

Furthermore he explained that their movement is disturbed by the other human beings wandering aimlessly in a world without a purpose. These poor souls are the result of man's tendency to rape and devour everything in his path and then hopelessly attempt to fill the void by breeding more monsters such as himself. It is utterly appaling to speculate how many species disappear every minute because of human arrogance, and it is even more unnerving to know that thanks to our tenacity and technical prowess we will likely get to watch everything around us die before we finally have nothing left to consume but ourselves.

These being the facts, it is remarkable that in the game the population does not appear overwhelmingly dense. Perhaps the author wishes to remind us that even a seemingly insignificant excess is a burden to human communities and nature alike.

Finally, it is worth noting that the game is written in Lisp, a language named after the popular speech impediment. This could perhaps be taken as a subtle allusion to the voice of sanity being drowned under the nonsense incessantly spouted by the economists and politicians who think they can build societies on the idea of perpetual growth in a world where the amount of resources remains a constant.

A brief examination of the source code, readily available thanks to the GNU General Public License v.3, reveals that the entire game world is internally represented as a hash table and passed around as an ever-evolving yet always complete whole. This design decision, arguably inconvenient from a practical standpoint but wonderfully expressive to the aesthetician, resonates with the beauty of the age-old doctrine that everything is One. There is solace to be found in the certainty that even after we are gone the Universe will still exist, renewing itself and eventually producing new life to undo our damage.

All this, of course, is irrelevant to the player.

\end{document}
